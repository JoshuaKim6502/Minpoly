\documentclass[a4paper]{article}
\usepackage{amsmath}
\usepackage{theorem}
\usepackage[all]{xy}
\usepackage{fancybox}
\usepackage{fancyhdr}
\usepackage[left=25mm,right=25mm,top=25mm,bottom=25mm]{geometry}

\theoremstyle{plain} %
{\theorembodyfont{\normalfont}
\newtheorem{Exa}{Example}}
%\newtheorem{Def}{Definition}
%\newtheorem{The}{Theorem}
%\newtheorem{Cor}{Corollary}

\def\mp{minimum polynomial }
\def\td{tracking/disturbance }
\def\dt{disturbance/tracking }

\begin{document}
\title{Minimum Polynomial \& \mbox{Disturbance/Tracking Poles}\thanks{This material is based on reference \cite{algebra1} and \cite{masten1}}}
\author{Edited by Joshua Kim\thanks{Mail:~\texttt{engr.joshua.kim@gmail.com}}}
\date{Oct. 1999} %
\maketitle %
%\normalsize

\thispagestyle{fancy} %
\pagestyle{fancy} %
\lhead{Minimum Polynomial \& Disturbance/Tracking Poles}  \rhead{Technical Note} %

%\tableofcontents

\section{Minimum Polynomial of A Square Matrix}
Let $A \! \neq \! 0$ be an $n$--square matrix over
$\mathcal{F}\footnote{$\cal F$ : Field- See Appendix.}$. ~~Since
$A \!\! \in \!\! \mathcal{M}_n(\mathcal{F})\footnote{$
\mathcal{M}_n(\mathcal{F})$ : Total matrix algebra (the set of all
n$\times$n matrices over $\cal F$) }$, the set $\{ I, A, A^2,
\ldots , A^{n^2} \}$ is linearly dependent and there exist scalars
$a_0,a_1,a_2,\ldots,a_{n^2}$ not all $0$ such that
\[
    \phi (A) = a_0 I + a_1 A + a_2 A^2 + \cdots + a_{n^2} A^{n^2}
    = 0
\]
In this section we shall be concerned with that
monic--polynomial\footnote{Monic Polynomial - See Appendix.}
$m(\lambda) \in \mathcal{F}[\lambda]\footnote{${\cal F}[x]$ : Set
of all polynomials in $x$ with coefficients in $\cal F$ }$ of
minimum degree such that $m(A)=0$. Clearly, either $m(\lambda) =
\phi(\lambda)$ or $m(\lambda)$ is a proper divisor of
$\phi(\lambda)$. In either case, $m(\lambda)$ will be called the
{\em minimum polynomial}~\cite[p.177]{algebra1} of $A$.

The most elementary procedure for obtaining the minimum polynomial
of $A \neq 0$ involves the following routine:

\begin{enumerate}
  \item If $A = a_0 I$, $a_0 \in \mathcal{F}$, then $m(\lambda)=\lambda - a_0$.
  \item If $A \neq aI$ for all $a\!\in\!\mathcal{F}$ but $A^2 =
  a_1A+a_0I$ with $a_0,a_1\in\mathcal{F}$, then
  $m(\lambda)={\lambda}^2-a_1\lambda-a_0$.
  \item If $A^2 \neq aA+bI$ ~for all $a,b\!\in\!\mathcal{F}$ but $A^3 =
  a_2A^2+a_1A+A_0I$ with $a_0,a_1,a_2\in\mathcal{F}$, then
  $m(\lambda)={\lambda}^3-a_2{\lambda}^2-a_1\lambda-a_0$
\end{enumerate}

and so on.

\begin{Exa}
Find the \mp of $A = \left[ \begin{array}{cc} 1 & 0
\\ 0 & 1 \end{array} \right]$. \\%
\vspace{0.5\baselineskip} %
Since $A-I=0$, the answer is $m(\lambda)=\lambda -1$.
\end{Exa}

\begin{Exa}
Find the \mp of $A = \left[ \begin{array}{cc} 0 & 1
\\ 0 & 0 \end{array} \right]$. \\%
\vspace{0.5\baselineskip} %
Since $A^2=0$, the answer is $m(\lambda)={\lambda}^2$.
\end{Exa}

\begin{Exa}\label{ex3}
Find the minimum polynomial of $A = \left[
\begin{array}{ccc} 1 & 2 & 2\\
2 & 1 & 2 \\ 2 & 2 & 1 \end{array} \right]$ over $Q$. \\ %

Since $A \neq a_0I$ for all $a_0 \in Q$, set%

\vspace{0.5\baselineskip} %

     $A^2 = \left[ \begin{array}{ccc}
            9 & 8 & 8 \\
            8 & 9 & 8 \\
            8 & 8 & 9
       \end{array} \right]
        = a_1 \! \left[ \begin{array}{ccc}
            1 & 2 & 2 \\
            2 & 1 & 2 \\
            2 & 2 & 1
       \end{array} \right]
       \; + \; a_0 \! \left[\begin{array}{ccc}
            1 & 0 & 0 \\
            0 & 1 & 0 \\
            0 & 0 & 1
       \end{array}\right]
       = \left[ \begin{array}{ccc}
            a_1\!+\!a_0 & 2a_1 & 2a_1 \\
            2a_1 & a_1\!+\!a_0 & 2a_1\\
            2a_1 & 2a_1 & a_1\!+\!a_0
       \end{array} \right]$

\vspace{0.5\baselineskip}

After checking every entry, we conclude that $A^2 = 4A + 5I$ and
the minimum polynomial is ${\lambda}^2 -4\lambda - 5$.

\end{Exa}

Foregoing example \ref{ex3} suggest that the constant term of the
minimum polynomial of $A \neq 0$ is different from $0$ if and only
if $A$ is non--singular. It can be used for computing the inverse
of a non--singular matrix follows.

\begin{Exa}
For $A$ of example \ref{ex3}, find the inverse $A^{-1}$. \\ %
Since $A^2-4A-5I = 0$ we have, after multiplying by $A^{-1}$,
$A-4I-5A^{-1} = 0$; %

\vspace{0.5\baselineskip}

hence, $A^{-1}=\frac{1}{5} (A-4I) = \frac{1}{5} \left[
\begin{array}{rrr} -3 & 2 & 2 \\ 2 & -3 & 2 \\ 2 &
2 & -3
\end{array} \right]$.

\end{Exa}

\section{Plant Model}
Consider multivariable linear time--invariant (LTI) plants, i.e.
it will be assumed that the plant to be controlled can be
described by the following model: %

\begin{align}\label{pmodel}
  \dot{x} &= Ax + Bu + E\omega \nonumber \\%
  y &= Cx + Du + F\omega \\ %
  y_m &= C_mx + D_mu + F_m\omega \nonumber \\%
  e &= y - y_{ref} \nonumber
\end{align}

where $u$ are the control inputs, $x$ is the state, $y$ are the
outputs to be regulated, $y_m$ are the measurable outputs,
$\omega$ are the disturbances, $y_{ref}$ are the tracking signals
and $e$ are the errors in the system. It will be assumed that the
plant has $m$ control inputs, $n$ states, $r$ outputs to be
regulated, $r_m$ measurable outputs, $\Omega$ disturbances, $r$
tracking signals and $r$ errors. Thus here the variables
$u,x,y,y_m,\omega,y_{ref},e$ are actually column vectors with
dimensions as follows:

\begin{align}
u &\in R^m \nonumber \\ %
x &\in R^n \nonumber \\%
y &\in R^r \nonumber \\ %
y_m &\in R^{r_m} \nonumber \\ %
\omega &\in R^{\Omega} \nonumber \\ %
y_{ref} &\in R^r \nonumber \\%
e &\in R^r \nonumber
\end{align}

\section{Class of controllers}
In general, in order to modify the behavior of the plant
(\ref{pmodel}) to solve the servomechanism problem, it is
necessary to apply a controller to it. A controller, in general,
is a dynamic device which is applied to the plant, and has as its
input $y_m$ and $y_{ref}$ and for its output $u$. Many different
types of controllers are possible to apply; in this lession, we
shall only consider LTI controllers. The most general LTI
controller is described by:

\begin{align}\label{class-ctr}
 \dot{\xi} &= {\Lambda}_{0}\xi + {\Lambda}_1 \, y_m + {\Lambda}_2
 \, y_{ref} \\
 u &= K_0 \xi + K_1 \, y_m + K_2 \, y_{ref} \nonumber
\end{align}

which has as inputs the tracking signal $y_{ref}$ and the
measurable outputs $y_m$, and has as outputs the control signal
$u$. Here $\xi$ is a state variable of controller and
$K_0,K_1,K_2,\Lambda_0,\Lambda_1,\Lambda_2$ are constant gain
matrices associated with the controller.

\section{Class of Tracking/Disturbance Signals}
In classical control, two classes of signals which are widely used
to describe tracking signals $y_{ref}$ and disturbance signals
$\omega$ are the class of step and ramp type signals. Here a large
class of signals will be considered, namely any signal which can
arise from an unstable LTI system will be considered. In
particular, it will be assumed that the class of tracking signals
and disturbance signals is specified, and arise from:

\begin{alignat}{2}\label{class-tds}
\dot{z}_1 &= A_1 z_1 \;,\; z_1 \in R^{\overline{n_1}} &\qquad %
\dot{z}_2 &= A_2 z_2 \;,\; z_2 \in R^{\overline{n_2}} \nonumber \\%
\sigma_1 &= C_1 z_1  &\qquad
\omega &= C_2 z_2 \\%
y_{ref} &= G \, \sigma_1 \nonumber %
\end{alignat}

for some initial conditions $z_1(0), z_2(0)$, where for
non--redundancy we assume that ($C_1, A_1$) and ($C_2, A_2$) are
observable, and that rank $G$ = dim($\sigma_1$). For
non--triviality, we assume that the signals $y_{ref}, \omega$ do
not decay to zero, i.e. we assume that $Re(\lambda_{1}^i)\geq 0 \,
, \,i=1,2,\ldots,\overline{n_1} \, , \; Re(\lambda_{2}^i)\geq 0 \,
, \, i=1,2,\ldots,\overline{n_2} \;$ where $\{ \lambda_{1}^i,\,
i=1,2,\ldots, \overline{n_1} \} \:,\: \{ \lambda_{2}^i,
\,i=1,2,\ldots, \overline{n_2} \}$ denote the eigenvalues of $A_1,
A_2$ respectively. this class of \td signals includes most classes
of signals which occur in industrial systems, e.g. constant, ramp,
polynomial, sinusoidal, polynomial--sinusoidal,
exponential--polynomial signals.\\

The above equations (\ref{pmodel}),(\ref{class-ctr}), and
(\ref{class-tds}) are very important for this lession since they
define the plant model, the controller, and the tracking and
disturbance signals. These models form the fundamental
mathematical building blocks for the material which we will study
in this lession. Figure \ref{model-ls} illustrates the
relationship between these mathematical models.

\begin{figure}[h]
\xymatrix{ & & \\%
    \fbox{\begin{Beqnarray*} % Tracking Signals
            \dot{z}_1 &=& A_1 z_1 \\ %
            \sigma_1 &=& C_1 z_1 \\ %
            y_{ref} &=& G \, \sigma_1
         \end{Beqnarray*}
    } \ar@/^1pc/[dr]^-{y_{ref}} \ar@/_4pc/[ddr]^-{y_{ref}} \ar@{}[u]|><<<{\txt{Tracking Signals}} & &
    \ovalbox{\begin{Beqnarray*} % Disturbances to the Plant
            \dot{z}_2 &=& A_2 z_2 \\ %
            \omega &=& C_2 z_2 %
         \end{Beqnarray*}
    } \ar@/_2pc/[dl]^-{\omega} \ar@{}[u]|><<<{\txt{Disturbances to the Plant}} \\   %
%
  &
    \shadowbox{\begin{Beqnarray*} % Plant
                \dot{x} &=& A \, x + B \, u + E \, \omega \\ %
                y &=& C \, x + D \, u + F \, \omega \\ %
                y_m &=& C_m \, x + D_m \, u + F_m \, \omega \\%
                e &=& y_{ref} - y
               \end{Beqnarray*}
    }\ar[r] \ar@{}[u]|><<<{\txt{Plant}}
  &
    \txt{Outputs\\$y , y_m$}  \\% Outputs
%
  & \fbox{\begin{Beqnarray*} % Controller
            \dot{\xi} &=& \Lambda_0 \, \xi + \Lambda_1 \, y_m + \Lambda_2 \, y_{ref} \\ %
            u &=& K_0 \, y_m + K_1 \, \xi + K_2 \, y_{ref} %
          \end{Beqnarray*}
    }\ar@{}[d]|><<<{\txt{Controller}} \ar[u]_-{u}
  & \\
  & &
} %
\caption{Mathematical model for linear system} %
\label{model-ls}
\end{figure}


Given (\ref{class-tds}), let $\{\lambda_1, \lambda_2, \ldots ,
\lambda_q \}$ be the zeros of the least common multiple of the \mp
of $A_1$ and \mp of $A_2$ (multiplicities repeated), and call

\begin{equation}
  \Lambda := \{\lambda_1, \lambda_2, \ldots ,\lambda_q \}
\end{equation}

the {\em \dt} poles\cite[pp.95--96]{masten1} of (\ref{class-tds}).\\%

{\em Note}: Often the class of \dt poles $\{\lambda_1, \lambda_2,
\ldots ,\lambda_q \}$ is used, rather than (\ref{class-tds}), to
describe the class of disturbance/reference input signals being
considered, e.g.

\begin{tabbing}
~~~\=$\bullet$ the class of constant signals is denoted by
~~~~~~~~~~~~~~~\=$\Lambda = \{0\}$ \\%
  \>$\bullet$ the class of ramp signals is denoted by  \>$\Lambda =
  \{0,0\}$ \\%
  \>$\bullet$ the class of sinusoidal signals ($sin \, \omega t$) is
  denoted by  \>$\Lambda = \{j\omega,-j\omega \}$ \\%
  \>$\bullet$ the class of exponential signals ($e^{\lambda t}$) is
  denoted by  \>$\Lambda = \{\lambda\}$ \\%
\end{tabbing}

\begin{Exa} Find the \dt poles of the system (\ref{class-tds})
given that

\begin{alignat}{2}
A_1 &= \left[ \begin{array}{cccc}
         1 & 0 & 0 & 0 \\
         0 & 1 & 0 & 0 \\
         0 & 0 & 0 & 1 \\
         0 & 0 & 0 & 0
      \end{array} \right]      &\qquad %
A_2 &= \left[ \begin{array}{rr}
         0 & \theta \\
         -\theta & 0
      \end{array} \right] \; , \; \theta \neq 0      \nonumber \\%
C_1 &= \left[ \begin{array}{cccc}
         1 & 0 & 1 & 0 \\
         0 & 1 & 0 & 0
       \end{array} \right]   &\qquad
C_2 &= \left[ \begin{array}{cc}
                1 & 0
       \end{array} \right]     \nonumber \\%
G &= \left[ \begin{array}{cc}
          1 & 0 \\
          0 & 1
     \end{array} \right] \nonumber
\end{alignat}

First, find the \mp $m_1(\lambda)$, $m_2(\lambda)$ of each $A_1$,
$A_2$ respectively.
\begin{alignat}{3}
{A_1}^3 &= {A_1}^2 &\qquad  &\rightarrow &\qquad m_1(\lambda) &= {\lambda}^2 (\lambda - 1) \nonumber \\%
{A_2}^2 &= -{\theta}^2 I &\qquad &\rightarrow  &\qquad
m_2(\lambda) &= {\lambda}^2+{\theta}^2 \nonumber
\end{alignat}

Thus the zeros of the least common multiple of the $m_1(\lambda)$
and $m_2(\lambda)$ is $\{\,0,\,0,\,1,\,j \theta, -j \theta \}$.

\end{Exa}

\vspace{\baselineskip}
\begin{center}
\sc \Large Appendix : Terminologies
\end{center}

\appendix

\section{Group}
A non-empty set $\cal G$ on which a binary operation $\circ$ is
defined is said to form a group with respect to this operation
provided, for arbitrary $a,b,c \in \cal G$, the following
properties hold: \cite[p.82]{algebra1}
\begin{enumerate}
 \item $(a \circ b) \circ c = a \circ (b \circ c)$ \hfill
 (associative law)
 \item There exists $u \in \cal G$ such that $a \circ u = u \circ
 a = a$ \\ \mbox{} \hfill (existence of identity element)
 \item For each $a \in \cal G$ there exists $a^{-1} \in \cal G$
 such that $a \circ a^{-1} = a^{-1} \circ a = u$ \\
 \mbox{} \hfill (existence of inverse)
\end{enumerate}

{\em Note}~1. The reader must not be confused by the use in 3. of
$a^{-1}$ to denote the inverse of $a$ under the operation $\circ$.
The notation is merely borrowed from that previously used in
connection with multiplication. Whenever the group operation is
addition, $a^{-1}$ is to be interpreted as the additive inverse
$-a$.\\

{\em Note}~2. A group is called {\em abelian}~ if the group
operation is commutative.

\section{Ring}
A non--empty set $\cal R$ is said to form a {\em ring} with
respect to the binary operations addition (+) and multiplication
($\bullet$) provided, for arbitrary $a,b,c \in \cal R$, the
following properties hold:  \cite[p.101]{algebra1}
\begin{enumerate}
  \item $(a+b)+c = a+(b+c)$ \hfill (associative law of addition)
  \item $a+b = b+a$ \hfill (commutative law of addition)
  \item There exists $z \in \cal R$ such that $a+z=a$.\\ \mbox{} \hfill
  (existance of an additive identity (zero))
  \item For each $a \in \cal R$ there exists $-a \in \cal R$ such
  that $a+(-a)=z$.\\ \mbox{} \hfill (existence of additive inverses)
  \item $(a \cdot b)\cdot c = a \cdot (b \cdot c)$ \hfill
  (associative law of multiplication)
  \item $a(b+c) = a \cdot b + a \cdot c$ \hfill (distributive
  laws)
\end{enumerate}

\section{Field}
A ring $\cal F$ whose non--zero elements form an abelian
multiplicative group is called {\em field}. \cite[p.118]{algebra1}

\section{Monic Polynomial}
Let $\cal R$ be a ring with unity $u$. Any polynomial $\alpha(x)$
of degree $m$ over $\cal R$ with reading coefficient $u$, the
unity of $\cal R$, will be called {\em
monic}.\cite[p.126]{algebra1}
\begin{Exa}
 The polynomials $1,\, x+3 ,\, \text{and~} x^2-5x+4$ ~are monic
 while $2x^2-x+5$ is not a monic polynomial aver $I$
\end{Exa}

\vspace{4\baselineskip}

\bibliography{mylib}
\bibliographystyle{ieeetr}

\end{document}
